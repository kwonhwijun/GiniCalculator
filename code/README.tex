\documentclass{article}

\usepackage{kotex}

\begin{document}
\section{프로젝트 개요}
현재 프로젝트는 대한민국 아파트 실거래가 데이터에 대한 하위지역 단위별 지니계수를 계산합니다.

\section{설치 및 환경 구성}
\subsection{개발 환경/의존성}
\begin{itemize}
    \item Python 버전: 3.9+
    \item 주요 라이브러리: pandas, numpy, yaml 등
\end{itemize}

\subsection{폴더 구조}
\begin{verbatim}
project/
├── data/
├── src/
│   ├── election_processor.py
│   ├── preprocess.py
│   └── ...
├── main.py
├── config.yaml
└── requirements.txt
\end{verbatim}

\section{사용 방법}
\subsection{실행 방법}
프로젝트를 실행하기 전, \texttt{config.yaml} 파일을 확인하고 필요한 설정(DB 경로, 매핑 폴더 등)을 기입합니다.

\section{주요 모듈 설명}
\subsection{election\_processor.py}

\begin{itemize}
    \item \textbf{역할}: 전체 파이프라인(전처리, 매핑, 지니계수 계산, 결과 저장)을 총괄
    \item \textbf{주요 함수}:
    \begin{itemize}
        \item \texttt{process\_election\_data()}:
              단일 선거 데이터 처리
        \item \texttt{process\_and\_save\_all\_elections()}:
              여러 선거 일괄 처리
    \end{itemize}
\end{itemize}

\subsection{preprocess.py}
\begin{itemize}
    \item \textbf{역할}: 원본 데이터 전처리 (결측치, 형 변환 등)
    \item \textbf{클래스/함수}:
    \begin{itemize}
        \item \texttt{DataProcessor}: 전처리 로직 담당
    \end{itemize}
\end{itemize}

\subsection{matching.py}
\begin{itemize}
    \item \textbf{역할}: 법정동/행정동 코드 매칭 담당
    \item \textbf{클래스/함수}:
    \begin{itemize}
        \item \texttt{Matcher}:
              \texttt{gen\_bdong()}, \texttt{conn\_code()} 등 시점별 코드 변환
    \end{itemize}
\end{itemize}
\end{document}